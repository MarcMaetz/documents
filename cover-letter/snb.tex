\documentclass[11pt,a4paper]{article}

\usepackage[margin=2.5cm]{geometry}
\usepackage[default]{lato}
\usepackage[T1]{fontenc}
\usepackage[utf8]{inputenc}
\usepackage[german]{babel}
\usepackage{setspace}
\usepackage{fancyhdr}

\setlength{\parindent}{0pt}
\setlength{\parskip}{6pt}
\pagenumbering{gobble}
\setlength{\headheight}{14pt}

\newcommand{\contactline}{%
  {\fontsize{10}{12}\selectfont
  Marc Maetz \textbar{} Langfurren 32 \textbar{} 8057 Zürich \textbar{} marc.maetz@proton.me \textbar{} +41 77 434 15 25}%
}

\pagestyle{fancy}
\fancyhf{}
\fancyhead[C]{\contactline}
\fancyfoot[C]{\contactline}

\begin{document}

Schweizerische Nationalbank (SNB)\\
z.H. Frau Kuster\\
Börsenstrasse 15\\
8001 Zürich

\vspace{12pt}

Zürich, 20. Januar 2026

\vspace{12pt}

\textbf{Bewerbung um die Stelle als Senior Applikationsentwickler (80–100\%)}

\vspace{6pt}

Sehr geehrte Frau Kuster

Die Schweizerische Nationalbank (SNB) als „unabhängige Zentralbank, die mit ihrer Geldpolitik wesentlich zur Preisstabilität, Finanzstabilität und zum Funktionieren der Schweizer Volkswirtschaft beiträgt, hat grossen Eindruck auf mich gemacht.“ Es wäre mir eine Ehre, mit meiner Erfahrung in der Entwicklung und im Betrieb datengetriebener Anwendungen als Senior Applikationsentwickler in der Abteilung „Applikationen Statistik einen Beitrag zur Erfüllung des gesetzlichen Auftrags der SNB zu leisten.“ In meiner aktuellen Position als Software-Ingenieur im Finanzumfeld bin ich für Konzeption und agile Entwicklung datengetriebener Anwendungen verantwortlich. Dazu gehören insbesondere die effiziente Verarbeitung grosser Datenmengen, die Optimierung von Datenbankabfragen, die Skalierung von verteilten Services sowie der zuverlässige Betrieb produktiver Systeme mit geeignetem Monitoring und automatisierten Tests. Diese Erfahrungen bringe ich gezielt für den Aufbau und den zuverlässigen Betrieb leistungsfähiger Applikationen im Umfeld der SNB ein. Ein Schwerpunkt meiner Arbeit liegt in der Automatisierung von Entwicklungs- und Betriebsprozessen, wodurch die Systemstabilität verbessert und die Incident-Auflösungszeiten reduziert werden können. In meiner Funktion als stellvertretender Leiter eines kleinen Engineering-Teams konnte ich meine Führungsverantwortung, meine ausgeprägte Organisationsfähigkeit sowie meine Fähigkeit zur koordinierten und verlässlichen Zusammenarbeit im Team unter Beweis stellen.

Mein Masterabschluss in Computational Science and Engineering an der ETH Zürich mit fundierten Statistik-Kenntnissen sowie meine Erfahrung mit datengetriebenen Backends, relationalen Datenbanken und automatisierten Build-Test- und Betriebsprozessen ermöglichen mir einen raschen und effizienten Einstieg in die Applikationslandschaft der SNB.

Als französisch-schweizerischer Doppelbürger mit Deutsch und Französisch als Muttersprachen sowie sehr guten Englischkenntnissen bringe ich die für die SNB erforderlichen Sprachkenntnisse mit.

Sehr gerne würde ich im Rahmen eines persönlichen Gesprächs meine Motivation näher darlegen und Sie von meinen fachlichen Fähigkeiten für die Stelle als Senior Applikationsentwickler überzeugen.

\vspace{12pt}

Freundliche Grüsse

\vspace{18pt}

Marc Maetz

\end{document}
